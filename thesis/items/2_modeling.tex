\graphicspath{{./images/model}}
\section{Построение модели}

Целью данного этапа является построение модели <<демонстрационного стенда Умного дома>>, которая затем будет использована для определения оптимального расположения датчика температуры. В качестве платформы для моделирования был выбран COMSOL Multiphysics.


\subsection{Построение модели куба}

В начале работы было решено создать простую модель деревянного куба с ребром $1$ метр и толщиной стен $15$ сантиметров, а затем проанализировать изменение температуры внутри. Для создания физической модели использовался SolidWorks, а затем модель была импортирована в COMSOL. В COMSOL внутренность куба заполнили воздухом (куб воздуха с ребром $0.7$ м), материал стен выбран как Wood (pine).

Начальная температура стен и воздуха внутри $293$ К. Температура, действующая на внешнюю поверхность стен куба, задана по формуле:
\[300 + 50 \cdot \sin(\frac{\pi \cdot t[s]}{500}) K\]

Промоделлирован промежуток времени длиной $1000$ секунд с шагом $50$.
Были рассмотрены различные размеры сетки разбиения: Normal (Рис. \ref{cube-normal}), Fine (Рис. \ref{cube-fine}), Finer (Рис. \ref{cube-finer}) и Extra fine (Рис. \ref{cube-extra-fine}).

Ниже можно увидеть полученные результаты. На первой картинке изображена сама сетка, на второй и третьей температура в разрезе в момент 250 и 750 секунд соответственно.

\begin{figure}[H]
\includegraphics[width=0.3\textwidth]{cube/normal_mesh.png}\hfill
\includegraphics[width=0.3\textwidth]{cube/normal_250s.png}\hfill
\includegraphics[width=0.3\textwidth]{cube/normal_750s.png}\hfill
\caption{Normal}
\label{cube-normal}
\end{figure}

\begin{figure}[H]
\includegraphics[width=0.3\textwidth]{cube/fine_mesh.png}\hfill
\includegraphics[width=0.3\textwidth]{cube/fine_250s.png}\hfill
\includegraphics[width=0.3\textwidth]{cube/fine_750s.png}\hfill
\caption{Fine}
\label{cube-fine}
\end{figure}

\begin{figure}[H]
\includegraphics[width=0.3\textwidth]{cube/finer_mesh.png}\hfill
\includegraphics[width=0.3\textwidth]{cube/finer_250s.png}\hfill
\includegraphics[width=0.3\textwidth]{cube/finer_750s.png}\hfill
\caption{Finer}
\label{cube-finer}
\end{figure}

\begin{figure}[H]
\includegraphics[width=0.3\textwidth]{cube/extra_fine_mesh.png}\hfill
\includegraphics[width=0.3\textwidth]{cube/extra_fine_250s.png}\hfill
\includegraphics[width=0.3\textwidth]{cube/extra_fine_750s.png}\hfill
\caption{Extra fine}
\label{cube-extra-fine}
\end{figure}

Можно видеть, что у Normal и Fine сеток большая погрешность, тогда как начиная с Finer результат выглядит достаточно гладко. Время вычисления для Finer составило 20 секунд, для Extra fine - 80 секунд.\par
Была попытка запустить вычисления на сетке размером Extremely fine, но ожидаемое время было слишком большим и требовалось много памяти, поэтому было решено остановиться на Extra fine, как наиболее точной при приемлемых затратах

\newpage


\subsection{Построение модели помещения}

За основу было выбрано помещение, смоделированное в программе FreeCAD, экспортированное сначала в формат STL (но из-за особенности формата не подошло), затем в STEP (не удалось построить сетку разбиения из-за неточностей в геометрии помещения). 
В итоге было принято решение построить упрощенную модель сразу в COMSOL.\\
Помещение размером $10 \times 6 \times 3 \text{ м}^3$ с окном, дверью и внутренней стеной с проходом. 
Начальная температура комнаты $293K$, на все стены, окно и дверь снаружи задана температура $330K$.
Временной промежуток 1000 секунд.\\
Размер сетки Extra fine. 
Полученные результаты можно увидеть на изображениях снизу

\begin{figure}[H]
\includegraphics[width=0.5\textwidth]{smart_room/simple/150s.png}\hfill
\includegraphics[width=0.5\textwidth]{smart_room/simple/350s.png}
\caption{Temperature at 150s and 350s}
\end{figure}

\begin{figure}[H]
\includegraphics[width=0.5\textwidth]{smart_room/simple/550s.png}\hfill
\includegraphics[width=0.5\textwidth]{smart_room/simple/750s.png}
\caption{Temperature at 550s and 750s}
\end{figure}

\newpage


\subsection{Добавление солнечной радиации}

Следующим шагом стало добавление солнечной радиации.
Все внешние стены комнаты и потолок, а также внутренние стены, на которые могло светить солнце сквозь стекло, были подвержены тепловому излучению.\\
В качестве источника радиации выбрано солнце, географическое расположение помещения - город Москва, дата 14.03.2023, солнечное излучение $1000[W/m^2]$.
Смоделирован промежуток времени протяженностью 24 часа.\\
Полученные результаты ниже

\begin{figure}[H]
\includegraphics[width=0.3\textwidth]{smart_room/solar_radiation/12.png}\hfill
\includegraphics[width=0.3\textwidth]{smart_room/solar_radiation/14.png}\hfill
\includegraphics[width=0.3\textwidth]{smart_room/solar_radiation/16.png}\hfill
\includegraphics[width=0.5\textwidth]{smart_room/solar_radiation/18.png}\hfill
\includegraphics[width=0.5\textwidth]{smart_room/solar_radiation/20.png}
\caption{Temperature at 12, 14, 16, 18, 20 hours}
\end{figure}

На этих изображениях можно отметить нагрев части стены и пола, на которую светит солнце через окно

\newpage


\subsection{Моделирование с реальными данными температур}

Для моделирования погодных условий был использован датасет с температурами, использованный для прошлых исследований, за май-август 2020 года. В нем с шагом примерно 1 секунда записаны показания датчика, расположенного снаружи. Было решено промоделировать 4 дня с шагом в 5 минут. Сетка выбрана finer, как наиболее сбалансированная по времени подсчета и качеству.

\begin{figure}[H]
\includegraphics[width=\textwidth]{smart_room/real_temperature/temperature_plot.png}
\caption{Ambient temperature plot}
\end{figure}





































