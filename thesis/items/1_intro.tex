\specialsection{Введение}

При постановке задачи регулирования для систем отопления, кондиционирования и вентиляции эксплуатирующие службы руководствуются двумя конкурирующими критериями: экономичностью работы системы и комфортностью внутренней среды. По различным оценкам, от 50 до 70 процентов всей расходуемой энергии приходится на системы отопления, вентиляции и кондиционирования воздуха (ОВК). Таким образом, оптимизация потребления этого класса устройств, пусть даже на 5-10 процентов, повлечет за собой ощутимое снижение общего уровня расхода энергии. 

В основе данной работы лежит идея скомбинировать два основных подхода к моделированию микроклимата помещений: методы вычислительной термодинамики (CFD) и методы сетевых воздушных потоков (NAF). Точное решение задачи CFD на небольшом временном промежутке позволит смоделировать работу измерительного комплекса, оптимизировать вектор измеряемых величин, количество датчиков и их расположение в помещении. 

Эта задача решается численными методами, так как сбор данных эмпирическим путем займет много времени и ресурсов. В стандартных офисных зданиях число отдельных зон может доходить до нескольких сотен, поэтому расстановка датчиков и считывание показаний очень трудоемкий процесс. При этом, используя сгенерированную модель, мы можем определить значения на датчиках одновременно для всех интересующих расположений.



\newpage
\specialsection{Постановка задачи}

Целью является разработка метода построения упрощенной динамической модели характеристик воздуха в помещении, на основе которой может быть реализован последующий сбор реальных данных. 

Достижение поставленной цели делится на два этапа, численное моделирование и анализ полученных результатов.

Первоначальное моделирование проводится методами вычислительной гидрогазодинамики (CFD), которые основаны на решении нелинейных уравнений тепло-массопереноса (Навье-Стокса). Результаты численного моделирования становятся основой для второго, аналитического этапа.

На втором этапе определяются параметры усовершенствованных моделей динамики характеристик воздуха в данном помещении, а также оптимальная пространственная конфигурация для измерительных приборов.