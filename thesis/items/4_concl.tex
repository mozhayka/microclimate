\specialsection{Выводы}

В итоге был разработан метод, который помогает целесообразно расположить датчик в помещении. Хотя он и не дает глобальной оптимизации, но конструктивно каждый шаг нашей процедуры направлен на улучшение качества системы, поэтому можно говорить о субоптимальном результате. С его помощью можем перейти от более сложной модели CFD к более простой.

Тем не менее, есть еще ряд проблем, над которыми стоит подумать. Одной из них является время года. Например, зимой температура снаруже гораздо ниже, чем внутри. То есть полученная регрессия для летнего периода не будет работать, и придется перерасчитывать коэффициенты. В том числе, временной сдвиг может стать другим, также как и само расположение датчика. Одним из возможных решений может стать усредненный критерий для разных времен года. На самом деле, даже летом для дневного и ночного периода есть разница температур, из-за которой может оказаться так, что ночью расположение датчика одно, а днем другое.

Еще одним вопросом может быть выбор критерия. Возможно стоит также учитывать и усредненную температуру по всем точкам. Тем самым точка у окна, температура в которой наиболее близкая к внешней, может быть достаточно далека от средней температуры, а значит и значение в ней будет меньше влиять на остальные.

Проверить полученный метод на практике пока не удалось, так как для этого нужно оснастить помещение датчиками и провести замеры. Однако, в будущем планируется применение этой процедуры для оборудования испытательного полигона, который будет построен в будущем. То есть эффективность будет позже проверена реальными измерениями.

\newpage

\specialsection{Заключение}

В результате работы был разработан алгоритм построения упрощенной динамичесткой модели характеристик воздуха в помещении. С его помощью можно находить расположение датчика, показания в котором позволят улучшить собираемые данные, что повысит качество прогноза.

Была создана упрощенная модель <<Демонстрационного стенда Умного дома>>, в которой учитывается взаимодействие с внешней средой, а также солнечная радиация. Осуществлен переход от численной модели к более простой, пригодной для быстрого расчета.

Полученный алгоритм можно найти по ссылке на GitHub: \url{https://github.com/mozhayka/microclimate}. Модели помещения в COMSOL также можно найти по этой ссылке.

\newpage

\specialsection{Благодарность}

Автор выражает особую благодарность Юрию Михайловичу Рассадину за регулярное внимание и наставления к данной работе