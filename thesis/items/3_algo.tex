\graphicspath{{./images/algo}}
\section{Аналитическая часть}

Целью этого этапа является нахождение оптимального расположения датчика с помощью полученной модели.

\subsection{Экспортирование результата в таблицу}


Для построения таблицы температур было решено выбрать 27 точек (3x3x3).

\begin{figure}[H]
\includegraphics[width=\textwidth]{comsol/interesting_points_9x9x9.png}
\caption{27 points}
\end{figure}

Далее нужно уменьшить шаг с 1 часа до 5 минут. Сетка Normal с шагом 5 минут считалась 3 минуты, сетка Finer - 8 минут. 
Если выбрать сетку Coarse (более грубую, чем Normal), то расчет с шагом 30 секунд на протяжении суток занимает 11 минут.\\
Для выбора оптимального варианта по времени и качеству, нужно сравнить температуры в точках на шагах в 5 минут.
Для этого можно рассмотреть значения температур в некоторых точках при значении времени 86100 сек (5 минут до окончания расчета).
Для coarser значения в первых 5 точках:\\


\begin{table}[H]
\centering
\begin{tabular}{c|c|c|c|c|c}
\textbf{Time (s)} & \textbf{Point 1} & \textbf{Point 2} & ... & \textbf{Point 27} & \textbf{Ambient} \\ \hline
0                 & 20.0             & 19.4             &     & 20.8              & 17.9             \\
300               & 19.9             & 19.4             &     & 20.7              & 17.9             \\
600               & 19.9             & 19.4             &     & 20.6              & 17.9             \\
...               &                  &                  &     &                   &                  \\
345600            & 21.2             & 21.5             &     & 20.8              & 19.9            
\end{tabular}
\end{table}


\begin{table}[H]
\begin{tabular}{lllllll}
Mesh type  & Time  & (0.15, 0.15, 0.15) & (5, 0.15, 0.15) & (9.85, 0.15, 0.15) & (0.15, 3, 0.15) & (5, 3, 0.15) \\
Finer & 86100 & 307,57             & 305,54          & 296,95             & 308,44          & 297,94       \\
Coarse& 86100 & 306,17             & 299,38          & 296,72             & 308,11          & 298,29       \\ 
\end{tabular}
\end{table}
Можно видеть, что в некоторых точках значения могут отличаться на 5 градусов, что достаточно большая погрешность. Скорее всего связано это с тем, что внешняя часть стен была нагрета солнцем, поэтому имеет большую температуру, а из-за грубости сетки эта температура была передана и внутренней части, в которой находиться точка.\\
Поэтому варианты либо уменьшить сетку и пожертвовать временем (увеличить длительность вычислений или увеличить временной шаг), либо рассматривать точки на некотором расстоянии от стен

\newpage


\subsection{Критерий оптимальности точки}

\subsection{Предсказание температуры}
